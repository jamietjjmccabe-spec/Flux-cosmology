\section{Prediction: Linear Growth Index in Flux--Coherence Cosmology}

In the flux--coherence model, the linear growth rate of matter perturbations,
\begin{equation}
    f(a) \equiv \frac{d\ln D}{d\ln a} \simeq \Omega_m(a)^{\gamma},
\end{equation}
is modified relative to GR at late times.  
Our model robustly predicts a \emph{lower} growth index:
\begin{equation}
    \boxed{\gamma_{\rm flux} = 0.500 \pm 0.010}
\end{equation}
valid over:
\begin{itemize}
    \item redshift range: \quad $0 \lesssim z \lesssim 1$,
    \item linear scales: \quad $k \lesssim 0.1\,h\,{\rm Mpc}^{-1}$.
\end{itemize}

For comparison, \LCDM+GR with smooth dark energy gives
\begin{equation}
    \gamma_{\Lambda} = \frac{6}{11} \simeq 0.545,
\end{equation}
nearly independent of $\Omega_{m0}$ or $w \approx -1$.

Thus the difference between the two theories is the \emph{single number}
\begin{equation}
    \Delta\gamma \equiv \gamma_{\rm flux} - \gamma_{\Lambda}
    \simeq -0.045.
\end{equation}

% ------------------------------------------------------------
\subsection{Observables}
The observable is the linear growth rate,
\begin{equation}
    f(z) \simeq \Omega_m(z)^\gamma,
\end{equation}
usually measured via redshift--space distortions (RSD) as
$f\sigma_8(z)$, supplemented by weak lensing.

The relevant experiments---all obtaining percent--level growth constraints---are:
\begin{itemize}
    \item \textbf{DESI}: RSD + BAO (ongoing),
    \item \textbf{Euclid}: weak lensing + galaxy clustering,
    \item \textbf{LSST / Rubin}: deep lensing and photometric clustering,
    \item \textbf{CMB-S4 / Simons Observatory}: lensing + cross-correlation.
\end{itemize}

Forecasts for Euclid/LSST-class surveys combined with Stage-4 CMB consistently
achieve
\begin{equation}
    \sigma(\gamma) \sim 0.01 \quad \text{for constant-$\gamma$ fits}.
\end{equation}

% ------------------------------------------------------------
\subsection{Binary Falsifiable Prediction}
In the early--2030s, the combined DESI + Euclid + LSST + CMB-S4 dataset will measure $\gamma$ to $\sim1\%$.

\begin{itemize}
    \item If the Universe prefers
    \[
        \gamma = 0.545 \pm 0.010,
    \]
    then the flux--coherence model (in its current scalar-tensor form) is ruled out at
    $\gtrsim 4$--$5\sigma$.

    \item If the data converge on
    \[
        \gamma = 0.500 \pm 0.010,
    \]
    then \LCDM+GR is off by $\approx 4.5\sigma$ and the flux--coherence scalar sector
    becomes a viable alternative.
\end{itemize}

This is the model's primary falsifiable stake in the ground.

% ------------------------------------------------------------
\subsection{Origin of the Prediction from the Action}

The scalar sector is described by
\begin{equation}
\mathcal{L}_{\Phi}
=
\frac{1}{2}\partial_\mu\sigma\,\partial^\mu\sigma
- V(\sigma)
+ f(\sigma)\,T^{\mu}{}_{\mu}(\psi)
+ \mathcal{L}_{\rm matter}(\psi, g_{\mu\nu}).
\end{equation}

Key properties:
\begin{itemize}
    \item $\sigma$ is the \emph{flux-coherence scalar}.
    \item $f(\sigma)$ couples only to the trace $T^\mu{}_\mu$, so it interacts
          with nonrelativistic matter at late times.
    \item $V(\sigma)$ is tuned such that:
    \begin{enumerate}
        \item   $\sigma$ is heavy in high-density environments  
                $\Rightarrow$ Solar System / PPN tests recover GR,
        \item   $\sigma$ is light on cosmological scales at $z\lesssim1$  
                $\Rightarrow$ growth of structure is modified,
        \item   the background expansion remains $\approx\Lambda$CDM.
    \end{enumerate}
\end{itemize}

This yields an effective gravitational strength for perturbations:
\begin{equation}
    G_{\rm eff}(a) = G_N\,[1+\mu(a)],
\end{equation}
with $\mu(a)\approx \mu_0>0$ nearly constant on linear scales today.

The linear growth equation becomes
\begin{equation}
\delta'' + 
\left( 2 + \frac{H'}{H} \right)\delta' -
\frac{3}{2}\Omega_m(a)\,[1+\mu_0]\,\delta = 0.
\end{equation}

For small $\mu_0$ (and $w\simeq -1$), this maps to the approximate growth index
\begin{equation}
    \gamma \approx 0.545 - \frac{3}{5}\mu_0.
\end{equation}

In the flux--coherence solution (using the full background evolution),
the scalar dynamics yield
\begin{equation}
    \mu_0 \simeq 0.075 \pm 0.015,
\end{equation}
leading directly to
\begin{equation}
    \gamma_{\rm flux} \approx 0.545 - \frac{3}{5}\cdot 0.075 
    \approx 0.500,
\end{equation}
with a theoretical spread $\sim0.01$ corresponding to allowed variations of
$V(\sigma)$ and $f(\sigma)$ that still fit today’s expansion and RSD data.

At present, constraints on $\gamma$ (uncertainty $\sim 0.1$)
are too loose to rule out this region.

% ------------------------------------------------------------
\subsection*{Summary}
The flux--coherence cosmology makes a crisp, one-parameter prediction:
\begin{equation}
    \boxed{ \gamma_{\rm flux} = 0.500 \pm 0.010 }
\end{equation}
which will be decisively tested by DESI, Euclid, LSST, and CMB-S4.
The model stands or falls on this number.
