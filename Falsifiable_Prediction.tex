\documentclass[11pt]{article}
\usepackage{amsmath, amssymb}
\usepackage{bm}

\begin{document}

\section{Prediction: Linear Growth Index in Flux--Coherence Cosmology}

In the flux--coherence model, the linear growth rate of matter perturbations,
\begin{equation}
    f(a) \equiv \frac{d\ln D}{d\ln a} \simeq \Omega_m(a)^{\gamma},
\end{equation}
is modified relative to GR at late times.
Our model robustly predicts a \emph{lower} growth index:
\begin{equation}
    \boxed{\gamma_{\rm flux} = 0.500 \pm 0.010}
\end{equation}
valid over:
\begin{itemize}
    \item redshift range: \quad $0 \lesssim z \lesssim 1$,
    \item linear scales: \quad $k \lesssim 0.1\,h\,{\rm Mpc}^{-1}$.
\end{itemize}

For comparison, $\Lambda$CDM+GR with smooth dark energy gives
\begin{equation}
    \gamma_{\Lambda} = \frac{6}{11} \simeq 0.545,
\end{equation}
nearly independent of $\Omega_{m0}$ or $w \approx -1$.

Thus the difference between the two theories is the \emph{single} number
\begin{equation}
    \Delta\gamma \equiv \gamma_{\rm flux} - \gamma_{\Lambda}
    \simeq -0.045.
\end{equation}

% You can continue the rest of the section here...

\end{document}
